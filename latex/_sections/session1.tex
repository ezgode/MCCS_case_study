%!TEX root = ../main.tex
\setcounter{chapter}{0}
\chapter{Linearization, Discretization and Simulation}
In this session we are going to go through the process of linearizing, discretizing, and simulating our system.
Our aim is to clarify the practical similarities and differences between the continuous-time-nonlinear, continuous-time-linear, and discrete-time-linear models of the system.\par
%
In order to do that, we will implement the three mentioned models of the system in a single Simulink diagram, and will apply a certain sequence of control inputs in open loop to it. 
Different kinds of control sequences and linearization points will reveal differences between the studied versions of the model. 
%
\section{Provided files}
\begin{itemize}
	\setlength\itemsep{0em}
	\item \texttt{open\_loop\_experiments.slx}.
		A Simulink file with some of the blocks needed to simulate the system. 
		In this file you will implement the continuous time non-linear, continuous time linear, and discrete time linear model, which are to be simulated in parallel given a certain sequence of control inputs to be applied in open loop. 
	\item \texttt{ex1.m}.
		A class whose methods have to be completed.
		A description of every function to be completed is included within the file.
	\item \texttt{exercise1\_Linearization.m}.
		A Matlab script that sets up and runs the required simulations.
		Modify its content only if you want to run other experiments in addition to the ones proposed.
	\item \texttt{utilities.m}.
		A class that gathers auxiliary functions that will be used in the scripts provided during the case study sessions.
		There is no need in reviewing its content.
	\item \texttt{circle, path\_1, path\_2, path\_3}  mat files containing different paths to track that can be used in the experiments.
\end{itemize}		
%
\section{Exercises}
\paragraph{Linearization}
	\begin{itemize}
		\setlength\itemsep{0em}
		\item Obtain the analytical expression of the nominal trajectories considering that the nominal state should describe a perfect path-tracking situation.
		\item Linearize the system and obtain the analytical expressions of the matrices $A,B,C,D$ characterizing the continuous-time linear model of the system.
		\item Complete the method \texttt{getSystemParameters} which should return a column vector including the parameters of the system given the values %
		$\curvature = 1e^{-10}$, %
		$\carLength = 4$, %
		$\accDyn = .8$, %
		$\steerDyn = .8$, %
		$\alpha = 3$.
		\item Complete the method \texttt{getLinealModelArrays} which should return the matrices $A,B,C,D$ given the column-vector with the parameters values. 
	\end{itemize}
\paragraph{Discretization}	
	\begin{itemize}
		\setlength\itemsep{0em}
		\item Complete the method \texttt{getDiscreteLinearModel} which should return the matrices $\Phi, \Gamma$ describing the discrete-time linear model of the system, calculated following three different methods: Euler approximation, using the method which exploits the series representation of the matrix exponential (that is using the auxiliary array $\Psi$ as described in section 2.2.), and using the Matlab command \texttt{c2d}.
	\end{itemize}
\paragraph{Simulation}
	\begin{itemize}
		\setlength\itemsep{0em}
		\item Implement the non-linear model within the block \texttt{Non linear model - continuous time} in the Simulink file. 
		Note that the value of the path curvature $\kappa$ in the equations must be taken from the \texttt{u8} signal, instead of using the corresponding fixed parameter. 
		\item Complete the Simulink diagram so that it applies the same sequence of control inputs to the three models of the system. 
		Keep in mind that, the control inputs and states you receive/send from/to the "from/to workspace" blocks must be the absolute ones (i.e. $u$ and $x$) and the linear models require $\tilde{u}$ and return $\tilde{x}$.
		\item Complete the method \texttt{getWorkingTrajectory} which should return the nominal trajectory of the control inputs and states in a format that can be used by the 'from-workspace' Simulink blocks. 
		\item Complete the method \texttt{getInitialState} which should return the initial state $x(0)$ of the system and the initial state $\tilde{x}(0)$ of the linear models.
		\item Complete the method \texttt{getOpenLoopControlSignal} which should return the sequence of control inputs to be applied in open loop in a format that is compatible with the "from-workspace" Simulink blocks.
		The sequence of control inputs should be designed so that you are able to observe the differences between models.
		\item Simulate and verify the system functionality. 
	\end{itemize}	
\paragraph{Experiments}	
	\begin{itemize}
		\setlength\itemsep{0em}
		\item Does the discrete-time linear model simulate sufficiently well the continuous-time linear model of the system?
		What is then interesting about using the discrete-time linear model instead of the continuous-time linear model to design control strategies?  
		\item Do you observe differences between the linear and non-linear models? Why is that? 
		\item Try keeping the same control sequence and changing the curvature/speed reference value used to linearize the model. 
		Does it change your previous observation? why?
	\end{itemize}

			\if\showSolution1
				\ \newline
				\noindent\fcolorbox{matColor}{matColor}{
					\begin{minipage}{\textwidth}
						\scriptsize
						%
						In this section we discuss how to obtain $\overline{\mathbf{\state}}(t)$, which will allow us to linearize the system (that is, expressing the dynamic model in terms of the incremental states $\tilde{\mathbf{\state}}(t) = \overline{\mathbf{\state}}(t) - \mathbf{\state}(t)$.\par
						%
						Notice that, aiming at perfectly tracking the given path at a constant speed, the next quantities can be derived by definition:
						\begin{align}
							\wstat{1}(t) & = \speedRef t \rightarrow \dwstat{1} = \speedRef \label{eq:ws1}\\
							\wstat{2}(t) & = 0 \label{eq:ws2}\\
							\wstat{3}(t) & = 0 \label{eq:ws3}\\
							\wstat{4}(t) & = \speedRef \rightarrow \dwstat{4} = 0 \label{eq:ws4}\\
							\dwstat{5}(t) & = 0
						\end{align}
						%
						From \eqref{eq:ss3}, \eqref{eq:ws3} and \eqref{eq:ws1}, the next follows: 
						\begin{align}
							0 & = \frac{\wstat{4}}{\carLength}\tan(\wstat{5}) - \curv(\wstat{1})\dwstat{1} \nonumber =\\
							  & = \frac{\speedRef}{\carLength}\tan(\wstat{5}) - \curv(\speedRef t)\speedRef
						\end{align}
						that is 
						%
						\begin{equation}
							\wstat{5}(t) = \arctan( \curv(\speedRef t)\carLength ) \label{eq:ws5}
						\end{equation}
						Finally, the control input trajectory can be easily derived from \eqref{eq:ss4} and \eqref{eq:ss5}, together with \eqref{eq:ws4} and \eqref{eq:ws5}.
						\begin{align}
							\wcon{1}(t) & = \wstat{4} \\
							\wcon{2}(t) & = \wstat{5}
						\end{align}
						Notice that in the previous derivation, the curvature of the represents an important source of potential non-linearities. 
						Hence, for the sake of simplicity, we will consider from now on \emph{constant-curvature} paths.\par
						%
						%Moreover, the states representing the global position 
						% \begin{align}
						% 	\dwstat{3}(t) & = \frac{\wstat{7}(t)}{\carLength}\tan(\wstat{8}(t)) = \frac{\speedRef}{\carLength} \curv\carLength \to \wstat{3}(t) = \speedRef \curv t \\
						% 	%
						% 	\dwstat{1}(t) & = \wstat{7}(t)\cos(\wstat{3}(t)) = \speedRef \cos(\speedRef \curv t ) \to \wstat{1}(t) = \frac{\sin(\speedRef \curv t )}{\curv} \\
						% 	\dwstat{2}(t) & = \wstat{7}(t)\cos(\wstat{3}(t)) = \speedRef \sin(\speedRef \curv t ) \to \wstat{2}(t) = -\frac{\cos(\speedRef \curv t )}{\curv}
						% \end{align}
						Considering the previous simplification and \eqref{eq:ws1}\textendash\eqref{eq:ws8}As a result, the working state trajectory previously derived can be written as: 
						\begin{align}
							%\wstat{1}(t) & = \frac{\sin(\speedRef \curv t )}{\curv} + \wstat{1}(0)\\
							%\wstat{2}(t) & = -\frac{\cos(\speedRef \curv t )}{\curv} + \wstat{2}(0)\\
							%\wstat{3}(t) & = \speedRef \curv t + \wstat{3}(0)\\
							%
							\wstat{1}(t) & = \speedRef t 					\label{eq:wss1}\\
							\wstat{2}(t) & = 0 								\label{eq:wss2}\\
							\wstat{3}(t) & = 0 								\label{eq:wss3}\\
							\wstat{4}(t) & = \speedRef 						\label{eq:wss4}\\
							\wstat{5}(t) & = \arctan( \curv \carLength ),    \label{eq:wss5}
						\end{align}
						%
						and the control input reference as 
						\begin{align}
							\wcon{1}(t) & = \wstat{4} = \speedRef\label{eq:wc1}\\
							\wcon{2}(t) & = \wstat{5} = \arctan( \curv \carLength )\label{eq:wc2}
						\end{align}
					\end{minipage}
				}
			\fi	
			\if\showSolution1
				\scriptsize
				\ \newline\noindent\fcolorbox{matColor}{matColor}{
					\begin{minipage}{\textwidth}
						The relevant derivatives to linearize the systems are:
						\begin{align}
							%& \der{1}{\stat{3}} = -\stat{7}\sin(\stat{3}),%
							%	\quad \der{1}{\stat{7}} = \cos(\stat{3})\\
							%
							%& \der{2}{\stat{3}} = \stat{7}\cos(\stat{3}),%
							%	\quad \der{1}{\stat{7}} = \sin(\stat{3})\\
							%
							%& \der{3}{\stat{7}} = \frac{\tan(\stat{5})}{\carLength},%
							%	\quad \der{3}{\stat{5}} = \frac{\stat{7}}{\carLength\cos^2(\stat{5})}\\
							%
							& \der{1}{\stat{2}} = \frac{\stat{4}\curv\cos(\stat{3})}{(1-\curv\stat{2})^2},%
								\quad \der{1}{\stat{3}} = -\frac{\stat{4}\sin(\stat{3})}{1-\stat{2}\curv}%
								\quad \der{1}{\stat{4}} = \frac{\cos(\stat{3})}{1-\stat{2}\curv} \\
							%
							& \der{2}{\stat{3}} = \stat{4}\cos(\stat{3}),\quad \der{2}{\stat{4}} = \sin(\stat{3})\\
							%
							& \der{3}{\stat{2}} = -\frac{\curv^2\stat{4}\cos(\stat{3})}{(\curv\stat{2}-1)^2}, \quad \der{3}{\stat{3}} = -\frac{\curv\stat{4}\sin(\stat{3})}{\curv\stat{2} - 1}, \quad \der{3}{\stat{4}} = \frac{\tan(\stat{5})}{\carLength} -\frac{\curv\cos(\stat{3})}{1-\stat{2}\curv}, \quad \der{3}{\stat{5}} = \frac{\stat{4}}{\carLength\cos^2(\stat{5})}\\
							%
							& \der{4}{\stat{4}} = -\accDyn,\quad \der{4}{\con{2}} = \accDyn\\
							%
							& \der{5}{\stat{5}} = -\steerDyn,\quad \der{4}{\con{1}} = \steerDyn
						\end{align}
						%
						With this derivatives, and applying the Taylor expansion series, we can rewrite the set of dynamic equations in terms of deviations w.r.t. the nominal trajectory.
						%
						\begin{equation}
							\dot{\tilde{\mathbf{\state}}} = A \tilde{\mathbf{\state}} + B \tilde{\mathbf{\control}}
						\end{equation}
						%
						with
						%
						\begin{align}
							A &= \begin{bmatrix}
								%0 & 0 & 0 & 0 & \speedRef\curv & 0 & 1 & 0 \\ 
								%0 & 0 & 0 & 0 & \speedRef\curv & 0 & 1 & 0 \\ 
								%0 & 0 & 0 & 0 & \speedRef\curv & 0 & 1 & 0 \\ 
								%
								0 & \speedRef\curv & 0 & 1 & 0 \\ 
								0 & 0 & \speedRef & 0 & 0 \\ 
								0 & -\curv^2\speedRef & 0 & 0 & \frac{\speedRef(1+\curv^2\carLength^2)}{\carLength} \\ 
								0 & 0 & 0 & -\accDyn & 0 \\ 
								0 & 0 & 0 & 0 & -\steerDyn
							\end{bmatrix}\\
							%
							B & = \begin{bmatrix}
								0 & 0 \\
								0 & 0 \\
								0 & 0 \\
								0 & \accDyn \\
								\steerDyn & 0 
							\end{bmatrix}
						\end{align}
					\end{minipage}
					%
				}
			\fi


% 	\subsection{Nominal trajectory}
% 		To linearize the system given the state space vector defined in \eqref{eq:stateVector}, we need to calculate a nominal trajectory [section 3.1.1.], due to the fact that the nominal situation in our case would be having the vehicle perfectly following the path.
% 		Remember that we are assuming that the path is parametrized with respect to the path coordinate $\pathCoor = \stat{5}$, and the curvature is expressed as $\curvature$. 
% 		Moreover, in this exercise we will assume that all the states can me measured.\par
% 		%
% 		\paragraph{To do:}
% 		\begin{itemize}
% 			\setlength\itemsep{0em}
% 			\item Set the parameters of the system to $\curvature = 0.04$, $\carLength = 4$, $\accDyn = 0.7$, $\steerDyn = 0.5$, $\speed_n = 13$. 
% 			\item Obtain the nominal trajectory ($\overline{\mathbf{\state}}(t)$ and $\overline{\mathbf{\control}}(t)$) of the system knowing that we aim at perfectly tracking the path ($\ddistToPath = 0, \distToPath = 0, \dyawErr = 0, \yawErr = 0$) with a constant longitudinal speed $\speed_n$, and that the path curvature is assumed to be constant (i.e. the path to be followed will be a circle).
% 		\end{itemize}	
% 		\paragraph{To report:}  
% 		\begin{itemize}
% 			\setlength\itemsep{0em} 
% 			\item[1)] Final expression of the nominal trajectory. 


% 		\end{itemize}						
% 		%
% 		%
% 	\subsection{Linear state space representation}
% 		\paragraph{To do:} 
% 		\begin{itemize}
% 			\setlength\itemsep{0em}
% 			\item Linearize the system and calculate $A$ and $B$.\par
% 		\end{itemize}	
% 		\paragraph{To report:} 
% 		\begin{itemize}
% 			\setlength\itemsep{0em}
% 			\item[2)] The analytical form of the matrices $A$ and $B$.\par

% 		\end{itemize}			
% 		%
% 	\subsection{Discrete state space representation}
% 	 	\paragraph{To do:}
% 		\begin{itemize}
% 			\setlength\itemsep{0em}
% 			\item Calculate the discrete representation of the system using Euler approximation. 
% 			\item Calculate the discrete representation of the system programmatically as explained in Section 2.2.
% 			\item Calculate the discrete representation of the system using Matlab functions. 
% 		\end{itemize}
% 		\paragraph{To report:} 
% 		\begin{itemize}
% 			\setlength\itemsep{0em}
% 			\item[3)] The analytical discrete representation using Euler approximation. 
% 			\item[4)] The numerical differences between the matrices obtained with the three different approaches. 
% 		\end{itemize}
% %
% 	\subsection{Simulation}
% 		\paragraph{To do:} 
% 		\begin{itemize}
% 			\setlength\itemsep{0em}
% 			\item Complete the Simulink diagram in order to run simulations where a specific control sequence $\control(t)$ is applied in open loop to the three versions of the system.
% 			\item Generate the nominal trajectory.  
% 			\item Set initial conditions for $\state$, and $\tilde{\state}$.
% 			\item Generate a control sequence to be applied in open loop. 
% 			\item Now, change the curvature to $0.01$ keeping everything else as in the previous set of experiments.
% 				What do you observe? Considering that you still applying the same sequence of control actions (in open loop) are the simulated trajectories equal to the case where $\curv = 0$? How do you explain it?
% 		\end{itemize}	
% 		\paragraph{To report:} 
% 		\begin{itemize}
% 			\setlength\itemsep{0em}
% 			\item[5)] Screenshot of the final Simulink scheme. 
% 			\item[6)] Simulation results.
% 			\item[7)] Comment the differences you observe. 
% 					  Are the results what you were expecting? 
% 					  Which are the consequences of linearizing the system? 
% 		\end{itemize}		